\documentclass[11pt,letterpaper,dvips]{article}
 
 
\usepackage{latexsym}
\usepackage{fancybox}
\usepackage{graphicx}
\usepackage{amssymb}
\usepackage{color}
%\usepackage{amstex}
\usepackage{ulem}
\usepackage{float}
 
\pretolerance=10000
\textwidth=7.0in
\textheight=9.6in
\voffset = -0.3in
\topmargin=0.0in
\headheight=0.00in
\hoffset = -0.3in
\headsep=0.00in
\oddsidemargin=0in
\evensidemargin=0in
\parindent=2em
\parskip=1.5ex
 
\def\rexp{{\rm e}}
\def\katzt{\ket{\alpha, t_0=0; t}}
\def\katt{\ket{\alpha, t_0; t}}
\def\usc{\mathcal{U}}
\def\tsc{\mathcal{T}}
\special{papersize=8.5in,11in}
\newcommand{\nid}{\noindent}
\newcommand{\ew}{W_\lambda}
\def\aap{A \& A}
\def\aj{AJ}
\def\apj{ApJ}
\def\apjl{ApJL}
\def\apjs{ApJS}
\def\apjsupp{ApJS}
\def\araa{ARAA}
\def\mnras{MNRAS}
\def\nat{Nature}
\def\pasp{PASP}
\def\hangpara{\par\hangindent 25pt\noindent}
\def\Lya{Ly$\alpha$}
\def\lya{Ly$\alpha$}
\def\Lyb{Ly$\beta$ }
\def\lyb{Ly$\beta$ }
\def\Lyg{Ly$\gamma$ }
\def\Lyd{Ly$\delta$ }
\def\etal{et al. }
\def\kms{km~s$^{-1}$ }
\def\mkms{{\rm \; km\,s\,^{-1}}}
\def\bmkms{{\rm\bf \; km\,s\,^{-1}}}
\def\L{$\lambda$}
\def\s-1{s$^{-1}$}
\def\Hz-1{Hz$^{-1}$}
\def\pohf{\rm p_\ohf}
\def\prhf{\rm p_\rhf}
\def\ohf{\frac{1}{2}}
\def\nohf{\frac{-1}{2}}
\def\rhf{\frac{3}{2}}
\def\sohf{\rm s_\ohf}
\def\esp{\Delta E_{2s, 2p}}
\def\chal{\frac{\Delta \alpha}{\alpha}}
\def\ylab{y_{\rm lab}}
\def\yaY{y \; {\rm and} \; Y}
\def\dadz{\frac{1}{\alpha} |\frac{{\rm d} \alpha}{{\rm d} z}|}
\def\abdadz{\frac{1}{\alpha} \left|\frac{{\rm d} \alpha}{{\rm d} z}\right|}
\def\abdadt{\frac{1}{\alpha} \left|\frac{{\rm d} \alpha}{{\rm d} t}\right|}
\def\sci#1{{\; \times \; 10^{#1}}}
\def\yold{y_{\rm old}}
\def\und#1{{\rm \underline{#1}}}
\def\dlmb{\Delta \lambda}
\def\abchal{| \chal |}
\def\lbar{\bar\lambda}
\def\elc{\rm e^-}
\def\qnn#1{\eqno {\rm {#1}}}
\def\l1l2{\lambda_2 \; {\rm and} \; \lambda_1}
\def\id{\indent}
\def\wid{\id\id}
\def\rid{\id\id\id}
\def\uid{\id\id\id\id}
\def\imp{\, \Rightarrow \>}
\def\di{\partial}
\def\ba{{\bf a}}
\def\bx{{\bf x}}
\def\bv{{\bf v}}
\def\bu{{\bf u}}
\def\bq{{\bf q}}
\def\br{{\bf r}}
\def\bs{{\bf s}}
\def\bp{{\bf p}}
\def\bj{{\bf j}}
\def\bk{{\bf k}}
\def\bm{{\bf m}}
\def\bn{{\bf n}}
\def\bd{{\bf d}}
\def\bg{{\bf g}}
\def\by{{\bf y}}
\def\bB{{\bf B}}
\def\bE{{\bf E}}
\def\bD{{\bf D}}
\def\bH{{\bf H}}
\def\bA{{\bf A}}
\def\bI{{\bf I}}
\def\bJ{{\bf J}}
\def\bK{{\bf K}}
\def\bL{{\bf L}}
\def\bS{{\bf S}}
\def\bF{{\bf F}}
\def\bN{{\bf N}}
\def\bP{{\bf P}}
\def\bR{{\bf R}}
\def\bU{{\bf U}}
\def\bM{{\bf M}}
\def\bY{{\bf Y}}
\def\bOm{{\bf \Omega}}
\def\bom{{\bf \omega}}
\def\bell{{\bf \ell}}
\def\di{\partial}
\def\ddi#1#2{\frac{\di #1}{\di #2}}
\def\gmm{\sqrt{1 - \frac{v^2}{c^2}}}
\def\eng{\varepsilon}
\def\dd#1#2{\frac{d #1}{d #2}}
\def\dt#1{\frac{d #1}{dt}}
\def\ddt{\frac{d}{d t}}
\def\ddit{\frac{\di}{\di t}}
\def\arr{\to \quad}
\def\smm{\sum\limits}
\def\intl{\int\limits}
\def\ointl{\oint\limits}
\def\liml{\lim\limits}
\def\prodl{\prod\limits}
\def\grd{{\bf \nabla}}
\def\crl{{\bf \grd} \times}
\def\dv{{\bf \grd} \cdot}
\def\ec{\frac{e}{c} \,}
\def\oc{\frac{1}{c}}
\def\xht{{\; \rm \hat x}}
\def\yht{{\; \rm \hat y}}
\def\zht{{\; \rm \hat z}}
\def\rht{{\; \rm \hat r}}
\def\pht{{\; \rm \hat \phi}}
\def\iht{{\; \rm \hat i}}
\def\jht{{\; \rm \hat j}}
\def\kht{{\; \rm \hat k}}
\def\theht{{\; \rm \hat \theta}}
\def\nht{{\; \rm \hat n}}
\def\oht{{\; \rm \hat 1}}
\def\tht{{\; \rm \hat 2}}
\def\vht{{\; \rm \hat v}}
\def\thht{{\; \rm \hat 3}}
\def\Rht{{\; \rm \hat R}}
\def\cyc{\Omega_c}
\def\dl{{d {\bf \ell}}}
\def\da{{\rm d {\bf a}}}
\def\ds{{\rm d {\bf s}}}
\def\smt{{\smll \rm T}}
\def\opi#1{\frac{1}{#1 \pi}}
\def\strs{\underline {\underline {\rm T}}}
\def\dtr{\, d^3 r}
\def\dftr{d^3 r \,}
\def\dtp{\, d^3 p}
\def\dftp{d^3 p \,}
\def\abrr{|\br - \br'|}
\def\rr{(\br - \br')}
\def\orr#1{\frac{#1}{\abrr}}
\def\trr#1{\frac{#1}{\abrr^3}}
\def\drr{\delta \rr}
\def\Grn{G(\br, \br')}
\def\Ylm{Y_{\ell m}}
\def\snth{\sin\theta}
\def\csth{\cos\theta}
\def\invr#1{\frac{1}{#1}}
\def\fpc{\frac{4 \pi}{c}}
\def\lgnl{P_\ell (\csth)}
\def\lgnn#1{{P_{#1} (\csth)}}
\def\thph{\theta, \phi}
\def\eht1{{\rm \hat e_1}}
\def\ehtb{{\rm \hat e_2}}
\def\iot{i \omega t}
\def\kr{\bk \cdot \br}
\def\ele{{\rm e^-}}
\def\iwc{\frac{i \omega}{c}}
\def\iwt{i \omega t}
\def\plr{{\rm \hat \epsilon}}
\def\skp{\smallskip}
\def\mkp{\medskip}
\def\bkp{\bigskip}
\def\elc{\rm e^{-s}}
\def\bmx{\left[\matrix}
\def\Det{{\rm Det \,}}
\def\bra#1{{<\negmedspace#1|}}
\def\ket#1{{|#1\negmedspace>}}
\def\kalp{\ket{\alpha}}
\def\kbet{\ket{\beta}}
\def\balp{\bra{\alpha}}
\def\bbet{\bra{\beta}}
\def\inpr#1#2
\def\otpr#1#2{{|#1><#2|}}
\def\dgg{\dagger}
\def\dual{\Leftrightarrow}
\def\hbm{\frac{\hbar^2}{2 m}}
\def\hbu{\frac{\hbar^2}{2 \mu}}
\def\mbh{\frac{2 m}{\hbar^2}}
\def\ih{\frac{i}{\hbar}}
\def\kjm{\ket{j m}}
\def\bjm{\bra{j m}}
\def\bjmp{\bra{j' m'}}
\def\ylm{Y_{\ell m}}
\def\pof#1{(\frac{#1}{2})}
\def\tqk{T_q^k}
\def\ptm{\frac{p^2}{2 m}}
\def\psn{\psi_n}
\def\tph{\frac{2 \pi}{\hbar}}
\def\plr{{\hat \epsilon}}
\def\Veff{V_{\rm eff}}
\def\rhs{\rho^{(s)}}
\def\rhis{\rho_i^{(s)}}
\def\tilH{\tilde H}
\def\tpi{2 \pi i}
\def\prin{\intl_P}
\def\ltk{\left [ \,}
\def\ltp{\left ( \,}
\def\ltb{\left \{ \,}
\def\rtk{\, \right  ] }
\def\rtp{\, \right  ) }
\def\rtb{\, \right \} }
\def\dc{\tilde d}
\def\bskp{\bigskip}
\def\nV{\Omega}
\def\d3x{d^3 x}
\def\mtI{\b{\b I}}
\def\cmma{\;\;\; ,}
\def\perd{\;\;\; .}
\def\delv{\Delta v}
\def\eikr{{\rm e}^{i \bk \cdot \br}}
\def\rmikr{{\rm e}^{i \vec k \cdot \vec r}}
\def\rmnikr{{\rm e}^{-i \vec k \cdot \vec r}}
\def\eikx{{\rm e}^{i \bk \cdot \bx}}
\def\muo{\mu_0}
\def\epo{\epsilon_0}
\def\rme{{\rm e}}
\def\dnu{\Delta \nu_D}
\def\rAA{\AA \enskip}
\def\halp{{\rm H_\alpha}}
\def\hbet{{\rm H_\beta}}
\def\N#1{{N({\rm #1})}}
\def\ekz{\rme^{i ( k_z z - \omega t )}}
\def\cm#1{\, {\rm cm^{#1}}}
\def\bcm#1{\; {\rm \mathbf{cm^{#1}}}}
\def\rmK{\; \rm K}
\def\rmm#1{\; {\rm #1}}
\def\Nperp{N_\perp}
\def\spkr#1{{\big \bf {#1}}}
\def\inA{\quad $\bullet$ \quad}
\def\inB{\quad\quad $\diamond$ \quad}
\def\inC{\quad\quad\quad $\circ$ \quad}
\def\sigfc{\sigma_{f_c}}
\def\sech{\rm sech}
\def\blck{\color[named]{Black}}
\def\blue{\color[named]{Blue}}
\def\skyblue{\color[named]{SkyBlue}}
\def\red{\color[named]{Red}}
\def\black{\color[named]{Black}}
\def\green{\color[named]{Green}}
\def\white{\color[named]{White}}
\def\yellow{\color[named]{Yellow}}
\def\purple{\color[named]{Purple}}
\def\maroon{\color[named]{Maroon}}
\def\redv{\color[named]{RedViolet}}
\def\mahog{\color[named]{Mahogany}}
\def\prb{$\frac{\delta \rho}{\rho}$}
\def\trd{\frac{2}{3}}
\def\mdn{\bar \rho}
\def\ss{v_s^2}
\def\msol{M_\odot}
\def\W#1{{W({\rm #1})}}
\def\Nav{N_a (v)}
\def\f#1{{f_{\rm #1}}}
\def\opt{$d \tau / d z \; $}
\def\btau{\bar\tau (v_{pk}) / \sigma (\bar \tau)}
\def\Ipk{I(v_{pk})/I_{c}}
\def\vrot{v_{rot}}
\def\sphr{\sqrt{R^2 + Z^2}}
\def\mbf{\mathbf}
\def\mgsqa{\, {\rm mag} / \Box ''}
\newcommand{\ialii}{Al{\sc II}}
\newcommand{\iari}{Ar{\sc I}}
\newcommand{\ini}{N{\sc I}}
\newcommand{\ioi}{O{\sc I}}
\newcommand{\iovi}{O{\sc VI}}
\newcommand{\iovii}{O{\sc VII}}
\newcommand{\ioviii}{O{\sc VIII}}
\newcommand{\ifeii}{Fe{\sc II}}
\newcommand{\ifeiii}{Fe{\sc III}}
\newcommand{\ispii}{S{\sc II}}
\newcommand{\iznii}{Zn{\sc II}}
\newcommand{\icaii}{Ca{\sc II}}
\newcommand{\inai}{Na{\sc I}}
\newcommand{\icii}{C{\sc II}}
\newcommand{\iciv}{C{\sc IV}}
\newcommand{\ihii}{H{\sc II}}
\newcommand{\ihi}{H{\sc I}}
\newcommand{\isiii}{Si{\sc II}}
\newcommand{\isiiv}{Si{\sc IV}}
\newcommand{\ici}{C{\sc I}}
\newcommand{\dcost}{d \negmedspace \cos\theta}
\newcommand{\mnvhi}{n_{\rm H {\sc I}}}
\newcommand{\nvhi}{$n_{\rm H {\sc I}}$}
\newcommand{\rem}{{\rm e}}
\def\otpr#1#2{{|#1><#2|}}
\def\avgq#1{<\negmedspace#1\negmedspace>}
\newcommand{\negm}{\negmedspace}
\newcommand{\dfhf}{^2\negm D_{\frac{5}{2}}}
\newcommand{\negt}{\negthinspace}
\newcommand{\nalph}{N/$\alpha$}
\newcommand{\alphh}{$\alpha$/H}
\newcommand{\nhi}{$N_{\rm HI}$}
\newcommand{\mnhi}{N_{\rm HI}}

\renewcommand{\descriptionlabel}[1]%
    {\hspace{\labelsep}\textsf{#1}}

\newenvironment{Aitemize}{%
 \renewcommand{\labelitemi}{$\bullet \;$}%
	\begin{itemize}}{\end{itemize}}

\newenvironment{Renumerate}{%
 \renewcommand{\theenumi}{\Roman{enumi}}
 \renewcommand{\labelenumi}{\theenumi)}
	\begin{enumerate}}{\end{enumerate}}

\newenvironment{Aenumerate}{%
 \renewcommand{\theenumi}{\Alph{enumi}}
 \renewcommand{\labelenumi}{\theenumi.}
	\begin{enumerate}}{\end{enumerate}}

\newenvironment{dmnd}{%
 \renewcommand{\labelitemi}{$\diamond \quad$}%
	\begin{itemize}}{\end{itemize}}

\newenvironment{Penumerate}{%
 \renewcommand{\labelenumi}{(\theenumi)}
	\begin{enumerate}}{\end{enumerate}}



\newcommand{\nnn}{ccd\#\#\#}
 
\special{papersize=8.5in,11in}

\renewcommand{\theenumi}{\Roman{enumi}}

\begin{document}
%\pagestyle{empty}
 

\begin{center}
{{\huge \bf ESI (Echellette) Data Reduction [v0.3]}}
\end{center}

\begin{enumerate}

{\Large   \item Echellette Suggested Calibrations }
   \begin{itemize}
	\item five Bias (0s) frames
	\item five Pin Hole images (dome?)
	\item ten QTZ Flats (Dome preferred) \quad  ({\it each slit size})
	\item Arc Calib:  five CuAr (600s) + five HgXe \quad  ({\it each slit size})
	\item one Standard star \quad ({\it each slit size})
   \end{itemize}

{\Large   \item Pre-Reduction Setup }
   \begin{itemize}
	\item Computer Resources
	\begin{itemize}
	  \item $>$300 MHz processor
	  \item 5G disk space / night
	  \item 750M RAM (1G on Solaris)
	  \item Linux or Solaris
	\end{itemize}
	\item Solaris WARNINGS (These do not apply to Linux)
	\begin{itemize}
	  \item Solaris+IDL does not release memory until IDL is exited
	  \item Consider exiting IDL occasionally and monitor memory usage
	  \item IDL+Solaris freezes up on the CPU intermittently
	  \begin{itemize}
		\item Possibly bug in licensing software
		\item Forced to kill IDL process and restart
	  \end{itemize}
	\end{itemize}
	\item Software
	\begin{itemize}
	  \item gcc or cc
	  \item IDL v5.4
	  \item IDL packages
		\begin{enumerate}
		  \item {\bf djs} IDL package  (Schlegel)
		  \item {\bf idlspec2d} IDL package  (Sloan public)
		  \item {\bf coyote} IDL package  
		  \item {\bf astron} IDL package   (Goddard)
		  \item {\bf xidl} IDL package   (JXP)
		\end{enumerate}
	\end{itemize}
	\item Log sheets [somewhat optional]
	\item This cookbook
   \end{itemize}

{\Large  \item Initial Setup (Repeat for each night) }

  \begin{enumerate}

    \item Create a new directory for the night (e.g.\ 06aug02) and enter it
    \item Create a 'Raw/' directory and put all the raw data in it.  
    \item If the filenames do not read esi\#\#\#\#.fits, it is quite likely
	the code will be unhappy
    \item gzip the data (gzip *.fits)
    \item Create a 'pro/' directory and copy the files from 
	$\$$XIDL\_DIR/ESI/pro/redux into it 
	(e.g.\ {\it proc\_ech.pro, extract\_ech.pro}).
	{\bf These files can and should be used to run most of the steps that follow.}
    \item Launch idl in the directory above Raw/ 
    \item {\bf esi\_strct} :: Create the esi structure.  
	\begin{itemize}
	\item This structure organizes
	the entire night of data and is the most important file created.  The 
	structure is listed on the next page.  
	\item The routine creates a few things: (1) an IDL structure in memory with
	whatever name you choose (e.g.\ esi); 
	(2) the file 'esistrct.fits' which is a fits
	version of the structure; (3) the file 'esi.list' which is an ASCII
	version of the fits file which lists the values of some but not all tags.
	To view the structure outside of IDL, I 
	recommend the program 'fv' which I think stands for fitsview.  It allows
	you to examine binary fits tables.  
	\item If the EDIT keyword is set, a
	gui will launch which allows some editing of the esi structure 
	(see below). \\
         \quad Example: IDL$> \;$ {\bf esi\_strct}, esi, /MKDIR, [/EDIT] \\  
         \quad Example: IDL$> \;$ {\bf proc\_ech}, esi, /MKSTRCT \\  
         \quad Time   : 1s per image
	\end{itemize}

	\begin{center}
	{\Large {\bf esistrct}}
	\end{center}

	{\small
	\begin{tabular}{lcl}
	  \hline
	  Tag & Type & Comment \\
	  \hline
         frame & 0     & FRAME Number \\
         flg\_anly & 0   &  Analysis flag 0=Don't Analyse, 2 = bias sub, 4=scatt \\
         Obj & ' '       & Object Name \\
         obj\_id & 0L     & Obj ID \\
         type & ' '    & ObjTyp : OBJ,STD,DRK,ZRO,IFLT,DFLT,ARC,MSK,IMG,TWI \\
         slit & 0.    & Slit width \\
         exp & 0.d        & Exposure time \\
         imfilt & ' '     & Image Filter : U,B,V,R,I, C \\
         mode & 0L     & Mode:  0=Image, 1=LowD, 2=Echellette \\
         CCD & ' '     & CCD \\
         TEL & ' '    & Telescope \\
         namp & 0     & Number of Amps \\
         arclamp& 0 & Arc Lamps: (Binary flag)  1=CuAr, 2=Xe, 4=HgNe \\
         qtzlamp& 0 & QTZ Lamp: 0=off, 1=On \\
         rotmode& 0&  Rotation mode: 0=Stationary\\ 
         ccdspeed& ' ' &        CCD Speed \\
         gain & 0.     & Gain \\
         readno & 0.     & Read Noise \\
         date & 0.0d    & Date of Obs \\
         UT & ' '    & UT \\
         RA & ' '    & RA \\
         DEC & ' '    & DEC \\
         Equinox & 0.     & EQUINOX \\
         refordr& 0L &         Reference order (ECH only) \\
         rootpth & ' '    & Path of the Root \\
         img\_root & ' '   & Root name (usually in Raw directory) \\
         flg\_ov &  0     & OV FILE?  0=No, 1=Yes   \\
         img\_ov &  ' '   & Name of OV file (with directory) \\
         flg\_final & 0   & Final File? 0=No \\
         img\_final & ' '  & Name of Final img \\
         ystrt & 0L      & Column for initiating the trace \\
         arc\_fil & ' '   & Name of the Arc image file (fits) \\
         map\_fil & ' '   & Name of the Map image file (fits) \\
         flat\_fil & ' '   & Name of the Flat image file (fits) \\
         obj\_fil & ' '  &  Name of object structure (fits) \\
	  \hline
	\end{tabular}
	}
	

    \item {\bf esi\_editstrct} :: Modify the ESI structure.
	The previous step creates the structure and takes a guess at the 
	initial values of many of the tags based on the header card info.  It
	is difficult, however, to automate all of the values for the tags and
	therefore the user should carefully edit the structure.  For most of
	the important tags, one can use {\bf esi\_editstrct}.  The rest must
	be done from the command line by hand.
	The obvious tags to modify are: 
	 \begin{itemize}
	  \item Obj :: Object name (this propagates into the final spectra 
		and {\bf should have no spaces!})
	  \item flg\_anly :: Include in analysis (defaulted to 1 for yes)
	  \item type :: ARC, IFLT, TWI, OBJ, STD, etc.  
	  \item mode :: ECH, LWD, IMG 
	 \end{itemize}

         \quad Example: IDL$> \;$ {\bf esi\_editstrct}, esi \\  
         \quad Time   : User interaction

    \item Writing the ESI structure to disk:: In IDL you can modify the values
	of any of the tags.  You can then save the structure in fits form and 
	rewrite the ASCII file  with the routine {\bf esi\_wrstrct}. \\
         \quad Example: IDL$> \;$ {\bf esi\_wrstrct}, esi, FITS='name' \\
         \quad Time   : fast

    \item Reading the ESI structure from disk:: {\bf esi\_ar}
         \quad Example: IDL$> \;$ esi = {\bf esi\_ar()} \\
         \quad Example: IDL$> \;$ esi = {\bf esi\_ar('esi\_name.fits')} \\
         \quad Time   : fast

    \item {\bf esi\_echsetup} :: ECH Setup
	\begin{itemize}
	  \item This routine examines the esi structure and looks for calibration
	files associated with the various slit widths of your ECH files.  
	It groups together exposures with identical Obj name and ESI mode and
	sets the obj\_id tags accordingly.
	A summary of the ECH exposures is put in 'esiECH\_summ.txt'. \\
         \quad Example: IDL$> \;$ {\bf esi\_echsetup}, esi \\
         \quad Example: IDL$> \;$ {\bf proc\_ech}, esi, /SETUP \\
         \quad Time   : Fast
	\end{itemize}

  \end{enumerate}

{\Large    \item Create Bias (Zero) frame}
	\begin{itemize}
	  \item We have found that creating a Zero frame is preferred
	over an overscan subtraction.  If you have taken Bias (0s) frames,
	the process is straightforward.  If you haven't, you can copy the image
	(or make a soft link) 'BiasS.fits' from 
	$\$$XIDL\_DIR/ESI/CALIBS into the 'Bias/' directory.\\
         \quad Example: IDL$> \;$ {\bf esi\_mkzero}, esi \\
         \quad Example: IDL$> \;$ {\bf proc\_ech}, esi, /MKZERO \\
         \quad Time   : 3-5min 
	  \item To check the Bias frame : {\bf xatv}, 'Bias/BiasS.fits'
	\end{itemize}

{\Large  \item Pinhole Map}
  \begin{itemize}
	\item One may have taken a series of pinhole traces using the
	pinhole mask and the QTZ lamp to trace out the light path along
	the CCD.  If so, the following task can be used to fit the pinhole
	traces and create a MAP which is required for the data reduction.
	In lieu of your own pinhole images, you will need to copy the files
	(or make soft links)
	{\it hole\_fit.idl, ECH\_map.fits, img\_hole.fits} from 
	$\$$XIDL/ESI/CALIBS to Maps/.  {\bf At this point, I would recommend
	using the CALIBS files over creating new ones.}
	
	\item {\bf esi\_trcholes} :: This routine will combine all pinhole
	images (denoted by the 9.99 slit width) and then trace the pinholes.
	Output is 'Maps/img\_hole.fits' and 'Maps/hole\_fit.idl' \\
         \quad Example: IDL$> \;$ {\bf esi\_echtrcholes}, esi \\
         \quad Example: IDL$> \;$ {\bf proc\_ech}, esi, maskid, /MKMAP \\
         \quad Time   : $<1$min 
  \end{itemize}

{\Large  \item Process Flats }
  \begin{itemize}
	\item This routine processes your QTZ calibration frames to (1) trace
	the edges of each order, (2) calculate the gain mismatch between multiple
	amplifiers, and (3) correct pixel-to-pixel variations.  
	Dome flats show significantly less scattered light than the internal
	flats and should be used when possible.
	Use the /IFLAT switch to use internal flats.
	\item The first routine simply combines the flats.  The output is
	a bias subtracted, median flat placed in 'Flats/'.  If one does not specify
	the slit size, the program loops through all sizes. \\
         \quad Example: IDL$> \;$ {\bf esi\_echmkflat}, esi, [slit], [/IFLAT]\\
         \quad Example: IDL$> \;$ {\bf proc\_ech}, esi, [slit], /MKFLAT [/IFLAT] \\
         \quad Time   : 30s per flat 
	\item The next step is to subtract scattered light and normalize the flat.
	The scattered light is quite significant for Internal flats but also
	important for Dome flats.  This code requires the slit size 
	explicitly.  The routine also normalizes the flat and also matches the
	gain on the two halves of the amplified CCD.  The
	output is 'Flats/FlatECH{\it slit}N.fits' where slit = 50, 75, or 10.\\
         \quad Example: IDL$> \;$ {\bf esi\_echfltsct}, esi, slit, [/IFLAT]\\
         \quad Example: IDL$> \;$ {\bf proc\_ech}, esi, slit, /NRMFLAT, [/IFLAT] \\
         \quad Time   : 1min 
	\item Check the Flat: {\bf xatv}, 'Flats/FlatECH{\it slit}N.fits'
  \end{itemize}

{\Large  \item Arc Image }
  \begin{enumerate}
	\item This is the most important step.  The sky subtraction
	algorithm will only succeed with a good Arc Image.  It should be 
	emphasized that we only create one Arc Image per slit size and apply
	this to all of the exposures taken with that slit size.  The following
	routines are well automated but should be checked carefully if you are
	doing a 'final' reduction.  I think it is possible that one could use
	an Arc image from a previous run or the CALIBS database 
	but wouldn't recommend it just
	yet.  All of the following routines require a specific slit size.

	\item {\bf esi\_echmkarc} :: Process the Arcs.  
	This step bias subtracts
	and flat fields the arc images.  If there is an image taken with all
	the lamps on (arclamp = 7) then the code will take that.  Otherwise,
	it will add two Arcs together to make a CuAr+HgNe+Xe image.  Output
	is: 'Arcs/ArcECH\_{\it slit}.fits' \\
         \quad Example: IDL$> \;$ {\bf esi\_echmkarc}, esi, slit \\
         \quad Time   : 1-15min

	\item {\bf esi\_echfitarc} :: This routine extracts a 1D spectrum down the
	center of each order and determines the wavelength solution.  
	There are two levels of interatction with this routine.  The most
	interaction (not recommended) is to use /INTER which prompts the
	user to identify and fit the Arc lines.  I recommend using the
	option /PINTER which has the program attempt to identify a set of
	lines in each order.  The user than interactively fits the lines
	using the routine {\bf x\_identify} which calls {\bf x1dfit}.
	As long as your arc lines are within several 
	pixels of my solution, things ought to run smoothly.
	The output is an IDL file containing the polynomial fits for each
	order: 'Arcs/ArcECH\_{\it slit}fit.idl'.  The program also creates
	a ps file to examine the quality of fits.   One can input their
	own set of linelists or restrict the code to use only CuAr lines
	(/CUAR).
         \quad Example: IDL$> \;$ {\bf esi\_echfitarc}, esi, slit, /PINTER \\
         \quad Time   : 2min per order

	\item {\bf esi\_echtrcarc} :: This routine traces arc lines in each
	order.  It first straightens the curved orders,  but does not rectify
	the arc lines.  It simply traces them and then maps back to the 
	original data frame in the following step.  I recommend interactively
	choosing the lines the program uses to trace but on can also let
	the program choose (/AUTO).  Output is a series of files in the
	'Arcs/TRC' directory (1 per order) including ps files for rough 
	inspection.  \\
         \quad Example: IDL$> \;$ {\bf esi\_echtrcarc}, esi, slit, /AUTO \\
         \quad Time   : 2min  ($>$10 min interactively)

	\item {\bf esi\_echmkaimg} :: This routine takes the arc traces and
	uses them to create a full Arc Image in the data frame.
	The final wavelength image has vacuum wavelengths.
	Output is : 'Arcs/ArcECH\_{\it slit}IMG.fits'\\  
         \quad Example: IDL$> \;$ {\bf esi\_echmkaimg}, esi, slit, /CHK \\
         \quad Time   : 5min  \\
         \quad Check output with: {\bf xatv}, 'Arcs/ArcECH\_{\it slit}IMG.fits'
  \end{enumerate}

{\Large  \item Process the Standard}
  \begin{itemize}
	\item This step is not necessary but will help with
	tracing the data and fluxing.  Alternatively, one can grab the standard
	trace from the CALIBS directory.
	\item {\bf esi\_echtrcstd} :: Process the standard star (bias, flatten),
	centroid it using a small region in order 11 (this can be modified), 
	skysubtract, and then trace it through the 10 orders.
	The output are a processed image in 'Final/', an Obj structure in 
	'Extract' (see below), and the file 
	'Extract/STD\_ECH{\it slit}\_TRC.fits' which is the trace for the 
	standard star. \\
         \quad Example: IDL$> \;$ {\bf esi\_echtrstd}, esi, slit  \\
         \quad Example: IDL$> \;$ {\bf proc\_ech}, esi, slit, /PROCSTD \\
         \quad Time   : 10min (mainly sky sub)
  \end{itemize}

\end{enumerate}

\clearpage

\begin{center}
{\Huge {\bf \quad 0. Extraction}}
\end{center}

  \begin{itemize}
	\item The following routines all apply to a single object, i.e.\
	multiple exposures of that object will be reduced together.
	\item I suggest you use a .pro file like {\it extrct\_ech.pro} 
	to guide the process.  It provides a convenient way 
	of stepping through the reduction.
	\item Most of the following routines take the {\it esi} structure
	and {\it obj\_id}.
  \end{itemize}

\begin{enumerate}

{\Large  \item Process the Image}

  \begin{itemize}
	\item {\bf esi\_echproc} :: Bias subtract and flat field the Raw image.
	This routine takes the index number of the esi structure as input.
	This is the integer in the first column of the file 'esi.list'.
	One can subtract scattered light with the flag (/SUBSCAT), but I
	currently don't recommend it.  The contribution is small for almost all
	objects.
	Output is in 'Final/' and is a flattened flux and variance fits file
	(one gzipped fits file with two exentsions per image). \\
         \quad Example: IDL$> \;$ {\bf esi\_echproc}, esi, indx, [/SUBSCAT]  \\
         \quad Example: IDL$> \;$ {\bf extrct\_ech}, esi, obj\_id, [/SUBSCAT] \\
         \quad Time   : 1min per image
         \quad Check  : {\bf xatv}, 'Final/f\_esi\#.fits'


  \end{itemize}

{\Large  \item Identify the Object}

  \begin{enumerate}
	\item {\bf esi\_echfndobj} :: Allows the user to interactively identify
	the science objects (default is auto selection).   It smashes a
	region of the spectrum and finds all peaks containing 4
	continuous pixels with 3 sigma significance.  The region is taken
	to be from 5600-5610A in order 4L (physical 11) by default.  These
	can be changed via the keyword REFWV=fltarr(2) and REFORDR=\#L.
	If not interactive, the program chooses the brightest object within
	20 pixels of the center of the slit.  Set the keyword /INTER to 
	interactively identify the object and a an aperture to mask for
	sky subtraction (recommended).
	It then creates an object structure which
	stores the spectra of all objects identified in the slit.
	The IDL structure created is described below.

\clearpage

	\begin{center}
	  {\Large {\bf specobjstrct} }
	\end{center}
	{\small
	\begin{tabular}{lcl}
	  \hline
	  Tag & Type & Comment \\
	  \hline
         field &  ' ' &  Name of field \\
         slit\_id &  0L & Used to store the order number (0-9) corresponding to
				physical order (15-6)\\  
         obj\_id &  ' ' & ID value (a=primary, b-z=serendip, x=NG) \\
         flg\_anly &  0 &  0=No analysis\\ 
         exp &  0. & \\
         xcen &  0L &  Column where obj was id\\
         ycen &  0. & \\
         flg\_aper &  0 &  0=boxcar\\
         aper &  fltarr(2) &  Widths of aperture, 0/1 = bottom/top (pixels)\\
         skyrms &  0. &  RMS of sky fit\\
         trace &  fltarr(5000) &\\
         npix &  0L & \\
         wave &  fltarr(5000) & \\
         fx &  fltarr(5000) & \\
         var &  fltarr(5000) &    $<=0$  rejected pix\\
         flg\_flux & 0 &    0=f$_\lambda$, 1=f$_\nu$\\
         flux &  fltarr(5000) &    Fluxed data\\
         sig &  fltarr(5000) &    Err in fluxed data\\
         date &  0.0d \\
         UT &  ' ' \\
         img\_fil &  ' ' \\
         slit\_fil &  ' ' \\
         instr\_strct &  ' ' & e.g. wfccdstr fits file\\
	  \hline
	\end{tabular}
	}

	\item The routine also does a pseudo-trace to the object based on the
	paths of the pin-hole traces.  This trace is used to mask out the
	object for sky subtraction.  At present the code assumes that there
	are no serendipitous objects in the slit to mask.
	\item Output is 'Extract/Obj\_esi{\it \#\#}.fits' \\
         \quad Example: IDL$> \;$ {\bf esi\_echfndobj}, esi, obj\_id, [exp], 
		[REFORDR=, REFWV= /NOCLOB, /INTER, /CHK, /STD] \\
         \quad Example: IDL$> \;$ {\bf extrct\_ech}, esi, obj\_id, /FNDOBJ \\
         \quad Time   : fast 

	\item {\bf esi\_echcopyfnd} :: If desired, the user can copy the
	trace from another object using this routine.   The program takes
	the trace from the alternate object and allows the user to choose
	a different aperture for masking. The keyword CPYIDX is required
	and refers to the index number of the template object.  An offset
	can also be applied to this trace using OFFSET=.  \\
         \quad Example: IDL$> \;$ {\bf esi\_echcopyfnd}, esi, obj\_id, [exp], 
		CPYIDX=, [APER=, OFFSET=]
         \quad Time   : fast 
  \end{enumerate}

{\Large  \item Sky Subtraction}
  \begin{itemize}
	\item {\bf esi\_echskysub} ::
	This procedure subtracts the sky from each order, one by one for
	each exposure with a given {\it obj\_id}.  The main recipe adopts either
	(a) a series of low order POLYNOMIALS row by row for orders 0 to 5
	(physical orders 15 to 11) or (b) a 1D 
	Bspline fit to the wavelength, flux pairs for orders 6-9. 
	I have found that the polynomials work best in the low sky regions and
	the BSPLINE offers a mild improvement in the sky heavy orders.
	There are some fancy features added in to deal with
	bright sky lines which get switched off or reduced for short exposures.
	\item Current issues::  One current failing of the code is sky subtraction
	in order 11 where the amplifiers meet.  The ratio of the gain is calculated
	from the Flat image and this ratio is not constant in time.  This leads
	to some over subtraction of sky in some regions on occasion.	
	\item Options ::
	\begin{itemize}
	  \item BORDR= 5L ::  Sets the order (non-physical) where the Bspline 
		routine takes over.
	  \item ORDR= [0L, 9L] ::  Sets the first and last order for sky subtraction.
		This is useful for repeating sky subtraction in an order or two
		with improved parameters.
	  \item SKLFIL= ' ' ::  Input file for fine tuning skyline subtraction with
		the Bspline routine.  Entires have this format: \\
		\quad ordr \# wvmin wvmax flag \\
		where the flag = 1 places breakpoints at the center of each row
		and 2 places one at each end.  One usually turns on /NOVAC with this
		keyword
	  \item /NOVAC :: Does not apply a vacuum correction to the skyline info.
	  	This probably would only be used with SKLFIL 
	\end{itemize}
	\item Output is the sky subtracted 2D image appended to the unsubtracted
	flux and the variance 2D images (e.g.\ 'Final/f\_esi{\#}.fits') \\
         \quad Example: IDL$> \;$ {\bf esi\_echskysub}, esi, obj\_id \\
         \quad Example: IDL$> \;$ {\bf extrct\_ech}, esi, obj\_id, /SKYSUB \\
         \quad Time   : 10min per exposure \\
	\item Examine the final product (and get wavelength info) by:\\
         \quad IDL$> \;$ {\bf xatv}, 'Final/f\_esi{\#}.fits', getsky=2, 
		WVIMG='Arcs/ArcECH\_{slit}IMG.fits'
  \end{itemize}

{\Large  \item Trace OBJ}
  \begin{itemize}
	\item {\bf esi\_echtrcobj} ::
	This procedure carefully traces the science objects.  
	By default, it will combine multiple exposures to improve tracing.  
	It uses the Standard star as a guide and a crutch.  
	This new trace overwrites the one in the Obj structure. 
	Turn the keyword /FAINT on for faint objects.  This step can (should)
	be skipped if you have used {\bf esi\_echcopyfnd} for 
	the trace above. \\
         \quad Example: IDL$> \;$ {\bf esi\_echtrcobj}, esi, obj\_id, 
		[exp], /FAINT \\
         \quad Example: IDL$> \;$ {\bf extrct\_ech}, esi, obj\_id, 
		[exp], /TRCOBJ, /FAINT\\
         \quad Time   : fast \\
  \end{itemize}


{\Large  \item Extraction}
  \begin{enumerate}
	\item {\bf esi\_echextobj} ::
	This procedure performs a boxcar extraction of the object in each
	order.  It then rebins to a 1D spectrum using linear interpolation
	of the wavelength solution.  This is the only binning of the data
	during the entire reduction scheme.  All spectra are also shifted to
	zero heliocentric velocity and placed on a wavelength array with 
	33000 pixels, w0 = 3900. and constant velocity pixels at 10km/s.

	\item  All of the output is written to the Obj structure \\
         \quad Example: IDL$> \;$ {\bf esi\_echextobj}, esi, obj\_id \\
         \quad Example: IDL$> \;$ {\bf extrct\_ech}, esi, obj\_id, /EXTOBJ \\
         \quad Time   : 3min 
	\item Examine the final product: \\
         \quad IDL$> \;$ {\bf esi\_echspecplt}
  \end{enumerate}

{\Large  \item Combine Multiple Exposures (REQUIRED)}
  \begin{itemize}
	\item {\bf esi\_echcombspec} ::
	This procedure adds up multiple exposures of the same object, order
	by order, weighting by S/N after scaling each exposure to have the
	same flux as the first.  It appropriately deals with CR's.  For objects
	with more than 2 exposures, it clips 5 sigma outliers as well.  Even
	if you have only a single exposure, you need to run this routine prior
	to fluxing and collapsing to a 1D spectrum as it creates the final
	spectrum structure that is necessary for fluxing and coadding to 1D.
	\item Output is a fits file in 'FSpec' named 'Obja\_ech.fits' where
	Obj is the name in the Obj tag and 
	the letter 'a' signifies the primary science object.  In the future,
	I might allow multiple objects to be extracted and would label 
	them 'b', 'c'..
         \quad Example: IDL$> \;$ {\bf esi\_echcombspec}, esi, obj\_id \\
         \quad Time   : fast \\
  \end{itemize}

{\Large  \item Flux the Spectra}
  \begin{itemize}
	\item {\bf esi\_echflux} ::
	This procedure fluxes each order of the combined frame.  Generally,
	I'd suggest using the flux calibration that is defaulted 
	(i.e. ECH\_FLUX\#.fits in CALIBS).
	\item This program overwrites the fx array in the file created by
	{\bf esi\_echcombspec}.
         \quad Example: IDL$> \;$ {\bf esi\_echfluxfin}, esi, obj\_id \\
         \quad Time   : fast \\
	\item Examine the final product: \\
         \quad IDL$> \;$ {\bf esi\_echspecplt}, /fspec
  \end{itemize}


{\Large  \item Collapse to 1D}
  \begin{itemize}
	\item {\bf esi\_echcoaddfin} ::
	This procedure coadds the various orders into a 1D spectrum.  It does
	weight by S/N but is otherwise rather simple minded.  This includes the
	fact that various regions of the spectra are ignored, especially data
	in the lower left hand of the CCD. 
	\item Output is two fits files in 'FSpec' with names 'Obja\_F.fits' and
	'Obja\_E.fits'.
         \quad Example: IDL$> \;$ {\bf esi\_echcoaddfin}, esi, obj\_id \\
         \quad Time   : fast \\
	\item Examine the final product: \\
         \quad IDL$> \;$ {\bf x\_specplot}, 'FSpec/Name\_F.fits', 
		'FSpec/Name\_E.fits'
  \end{itemize}

\end{enumerate}

\clearpage

\noindent {\Large {\bf \quad 0. LWD Reduction}}

  \begin{itemize}
	\item The following routines setup and then proceed through the
	reduction of data taken in the LWD mode with ESI.
	\item I suggest you use a .pro files like 
	{\it proc\_lwd.pro,extrct\_lwd.pro} 
	to guide the process.  
	\item Most of the following routines take the {\it esi} structure,
	and the slit size or the {\it obj\_id}.
  \end{itemize}

\begin{enumerate}

{\Large    \item Create Bias (Zero) frame}
	\begin{itemize}
	  \item We have found that creating a Zero frame is preferred
	over an overscan subtraction.  If you have taken Bias (0s) frames,
	the process is straightforward.  If you haven't, you can copy the image
	'BiasS.fits' from $\$$XIDL/ESI/CALIBS into the 'Bias/' directory.\\
         \quad Example: IDL$> \;$ {\bf esi\_mkzero}, esi \\
         \quad Example: IDL$> \;$ {\bf proc\_lwd}, esi, /MKBIAS \\
         \quad Time   : 3-5min 
	  \item To check the Bias frame : {\bf xatv}, 'Bias/BiasS.fits'
	\end{itemize}

{\Large  \item {\bf esi\_lwdsetup} :: LWD Setup}
	\begin{itemize}
	  \item This routine examines the esi structure and looks for calibration
	files associated with the various slit widths of your LWD files.  
	It groups together exposures with identical Obj name and ESI mode and
	sets the obj\_id tags accordingly.
	A summary of the LWD exposures is put in 'esiLWD\_summ.txt'. \\
         \quad Example: IDL$> \;$ {\bf esi\_lwdsetup}, esi \\
         \quad Example: IDL$> \;$ {\bf proc\_lwd}, esi, /SETUP \\
         \quad Time   : Fast
	\end{itemize}

{\Large  \item Arc Image }
  \begin{enumerate}
	\item This is the most crucial step.  The sky subtraction
	algorithm will only succeed with a good Arc Image.  It should be 
	emphasized that we only create one Arc Image per slit size and apply
	this to all of the exposures taken with that slit size.  The following
	routines are well automated but should be checked carefully if you are
	doing a 'final' reduction.  I think it is possible that one could use
	an Arc image from a previous run but wouldn't recommend it just
	yet.  All of the following routines require a specific slit size.
	\item For LWD mode, only the HgNe and Xe lamps are useful.  Therefore
	we focus on these images (i.e.\ arclamp = 6).  
	\item {\bf esi\_lwdmkarc} :: Process the Arcs.  This step bias subtracts,
	trims and combines all of the Arc images.
        Note that we find it is best not to flat field the Arc, at least not
	with an unnormalized arc. Output is: 'Arcs/ArcLWD\_{\it slit}.fits' \\
         \quad Example: IDL$> \;$ {\bf esi\_lwdmkarc}, esi, slit \\
         \quad Time   : 2min

	\item {\bf esi\_lwdfitarc} :: This routine extracts a 1D spectrum along
	row 820 (default) of Arc image and determines the wavelength solution.  
	One can perform
	the solution interactively (/INTER), but I recommend letting the code
	run automatically.  As long as your arc lines are within 3 pixels of my 
	solution, things ought to run smoothly.
	The output is an IDL file containing the polynomial fits for each
	order: 'Arcs/AFIT\_LWD{\it slit}.fits'.  The program also creates
	a ps file to examine the quality of fits.\\
         \quad Example: IDL$> \;$ {\bf esi\_lwdfitarc}, esi, slit \\
         \quad Time   : 2min 

	\item {\bf esi\_lwdtrcarc} :: This routine traces arc lines throughout
	the image.  It will parse out the ends of traces where the algorithm
	tends to have trouble.
	Output is an anonymous structure written to the binary fits 
        file 'Arcs/ATRC\_LWD{\it slit}.fits'. \\
         \quad Example: IDL$> \;$ {\bf esi\_lwdtrcarc}, esi, slit \\
         \quad Time   : 2min  

	\item {\bf esi\_lwdmkaimg} :: This routine takes the arc traces and
	uses them to create a full Arc Image in the data frame.  The main
	routine performs a 2D polynomial surface fit to the traces from
	the previous step.  This is very expensive but needs to be only
	run once per night per slit width.
	Output is : 'Arcs/ArcECH\_{\it slit}IMG.fits'\\  
         \quad Example: IDL$> \;$ {\bf esi\_lwdmkaimg}, esi, slit /AUTO \\
         \quad Time   : 20min  \\
         \quad Check output with: {\bf xatv}, 'Arcs/ArcECH\_{\it slit}IMG.fits'
	\item To run all three steps at once, use:\\
         \quad IDL$> \;$ {\bf proc\_lwd}, esi, slit, /MKAIMG, [/INTER] \\
  \end{enumerate}

{\Large  \item Process Flats }
  \begin{itemize}
	\item This routine processes the QTZ calibration frames, producing
	a normalized flat useful to correcting pixel-to-pixel variations. 
	The routine bias subtracts and combines individual exposures.
	It then fits a high order bspline to a strip of the flat in order
	to normalize to unity.
	\item Output is 'Flats/FlatLWD\_{\it slit}.fits'
         \quad Example: IDL$> \;$ {\bf esi\_lwdmkflat}, esi, slit \\
         \quad Example: IDL$> \;$ {\bf proc\_lwd}, esi, slit, /MKFLAT \\
         \quad Time   : 5-10min
	\item Check the Flat: {\bf xatv}, 'Flats/FlatLWD\_{\it slit}.fits'
  \end{itemize}
\end{enumerate}

\end{document}
	

